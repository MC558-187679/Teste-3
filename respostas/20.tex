Dada um grafo $(G, \varomega)$, $s, t \in V[G]$ e um valor $k > 0$, mostre como usar Dijkstra para determinar o maior preço entre todos os caminhos de $s$ a $t$ de peso total no máximo $k$. Explique sua ideia sucintamente e escreva um pseudo-código para seu algoritmo. Você não  precisa provar que o algoritmo está correto, mas sua explicação deve ser clara o suficiente para eu me convencer disto. A complexidade do seu algoritmo deve ser $O(V+E + f(V+E))$. Note que você não precisa devolver o caminho, apenas o preço dele.

\begin{enumerate}[label={(\alph*)}]
    \item Nesta questão, todos os grafos são e devem ser representados por listas de adjacências. Além, não têm arestas de peso negativo.

    \item Você tem à disposição um algoritmo chamado Dijkstra que dado um grafo orientado ponderado $(G, \varomega)$ e um vértice $s \in V[G]$, devolve um vetor $d[~]$ indexado por $V[G]$ tal que $d[v] = \mathrm{dist}(s, v)$ para todo $v \in V[G]$. Este algoritmo é dado como uma caixa-preta, i.e., você não sabe como ele é implementado internamente.

    \item Suponha que Dijkstra tem complexidade de tempo $O(f(V+E))$ (uma função de $V$ e $E$). A complexidade exata não é importante, mas deve ser respeitada na solução (veja abaixo).
\end{enumerate}

\itemdsep

O algoritmo é baseado em três informações:

% \begin{minipage}{0.5\textwidth}
%     \begin{figure}[H]
%         \centering
%         \begin{tikzpicture}[
    auto, node distance=1cm and 1cm, on grid,
    semithick, inner sep=0pt,
    dot/.style={circle, color=black, fill=black, draw, black, text=black, minimum width = 4pt}
]
    \begin{pgfonlayer}{background}
        \node[dot, label = above left : s] at (0, 0) (s) {};
        \node[dot, label = above left : u] at (3, 0) (u) {};
        \node[dot, label = above right :v] at (4,1.5) (v) {};
        \node[dot, label = below left : t] at (5,-0.5) (t) {};

        \draw (s) -- (1,1) -- (2.5,2) -- (v);
        \draw (s) -- (1,-1) -- (2,-1) -- (2,0);
        \draw (2,-1) -- (3,-2) -- (4,-0.5) -- (u);
        \draw (1,1) -- (2,0) -- (u);
        \draw (4,-0.5) -- (t) -- (5.5,1) -- (v);
        \draw (5.5,1) -- (7,-1.5) -- (3,-2);
        \draw (7,-1.5) -- (t);
    \end{pgfonlayer}

    \begin{pgfonlayer}{foreground}
        \path (s) edge[-latex, bend right = 40, color=blue!60!black, opacity=0.7]
            node[below, font=\scriptsize] (ds)
                {$d_s[u] = \mathrm{dist}(s, u)$} (u);
        \path (v) edge[-latex, bend left = 40, color=blue!60!black, opacity=0.7]
            node[above right, font=\scriptsize] (dt)
                {$d_t[v] = \mathrm{dist}(v, t)$} (t);
        \path (u) edge[-latex, color=red!50!black]
            node[font=\scriptsize] {$\varomega(u v)$} (v);
    \end{pgfonlayer}

    \node[shape=ellipse, fill=white, fit=(ds), opacity=0.8, path fading=circle with fuzzy edge 15 percent, inner sep=0.5pt] {};
    \node[shape=ellipse, fill=white, fit=(dt), opacity=0.8, path fading=circle with fuzzy edge 15 percent, inner sep=0.5pt] {};
\end{tikzpicture}

%         \caption{x}
%         \label{fig:ex2}
%     \end{figure}
% \end{minipage}

\begin{enumerate}
    \item Como o preço de um caminho só depende da aresta de maior peso, não é preciso calcular o preço de todos os caminhos. No caso, a solução será apenas o maior peso $\varomega(u, v)$ dentre todas arestas $u v$ que aparecem em um caminho válido (caminhos de $s$ a $t$ com peso total $\leq k$). Isto é:
    \[
        \max_{C\text{ é válido}}\set{\text{preço}(C)}
        = \max_{C\text{ é válido}} \set{\max_{u v \in C}\{\varomega(u, v)\}}
        = \max_{u v\text{ é parte de um cam. válido}} \set{\varomega(u, v)}
    \]

    \item Para decidir se uma aresta $u v$ é parte de um caminho válido, basta checar se \\ $\dist(s, u) + \varomega(u, v) + \dist(v, t) \leq k$. Nesse caso, $\dist(s, u) + \varomega(u, v) + \dist(v, t)$ é o peso total do menor caminho de $s$ a $t$ que passa por $u v$. Se esse caminho não for válido, então nenhum outro que passa por $u v$ será e, portanto, $u v$ não deve ser considerada. Caso contrário, temos um caminho válido com $u v$.

    A ideia está exemplificada na \cref{fig:ex2}.

    \item $\dist(s, u)$ pode ser facilmente encontrada com o resultado de $\proc{Dijkstra}$. Para $\dist(v, t)$, no entanto, podemos aplicar o algoritmo de Dijkstra no grafo transposto $G^\intercal$, de modo que o resultado é $\dist_{G^\intercal}(t, v) = \dist_{G}(v, t)$ para todo $v \in V[G] = V[G^\intercal]$. A função $\varomega^*$ associa para cada aresta $v u \in E[G^\intercal]$ o mesmo peso $\varomega$ da aresta equivalente $v u \in E[G]$ no grafo original.
\end{enumerate}

\begin{minipage}{0.66\textwidth}
    \begin{codebox}
        \Procname{$\proc{Maior-Preço}(G, \varomega, s, t, k)$}

        \li $d_s[~] \Recebe \proc{Dijkstra}(G, \varomega, s)$
        \li \Seja $\varomega^*$ definida por $\varomega^*(u v) = \varomega(v u)$
        \li $d_t[~] \Recebe \proc{Dijkstra}(G^\intercal, \varomega^*, t)$
        \li
        \li $\id{preco} \Recebe -\infty$
        \li \Para \Cada vértice $u \in V[G]$ \Faca
            \Do
        \li     \Para \Cada vértice adjcente $v \in \Adj[u]$ \Faca
                \Do
        \li         \Se $d_s[u] + \varomega(u v) + d_t[v] \leq k$
                    \Do
        \li             \Entao $\id{preco} \Recebe \max\{\id{preco}, \varomega(u, v)\}$
                    \End
                \End
            \End
        \li \Devolva $\id{preco}$
    \end{codebox}
\end{minipage}%
\begin{minipage}{0.28\textwidth}
    \begin{figure}[H]
        \centering
        \begin{tikzpicture}[
    auto, node distance=1cm and 1cm, on grid,
    semithick, inner sep=0pt,
    dot/.style={circle, color=black, fill=black, draw, black, text=black, minimum width = 4pt}
]
    \begin{pgfonlayer}{background}
        \node[dot, label = above left : s] at (0, 0) (s) {};
        \node[dot, label = above left : u] at (3, 0) (u) {};
        \node[dot, label = above right :v] at (4,1.5) (v) {};
        \node[dot, label = below left : t] at (5,-0.5) (t) {};

        \draw (s) -- (1,1) -- (2.5,2) -- (v);
        \draw (s) -- (1,-1) -- (2,-1) -- (2,0);
        \draw (2,-1) -- (3,-2) -- (4,-0.5) -- (u);
        \draw (1,1) -- (2,0) -- (u);
        \draw (4,-0.5) -- (t) -- (5.5,1) -- (v);
        \draw (5.5,1) -- (7,-1.5) -- (3,-2);
        \draw (7,-1.5) -- (t);
    \end{pgfonlayer}

    \begin{pgfonlayer}{foreground}
        \path (s) edge[-latex, bend right = 40, color=blue!60!black, opacity=0.7]
            node[below, font=\scriptsize] (ds)
                {$d_s[u] = \mathrm{dist}(s, u)$} (u);
        \path (v) edge[-latex, bend left = 40, color=blue!60!black, opacity=0.7]
            node[above right, font=\scriptsize] (dt)
                {$d_t[v] = \mathrm{dist}(v, t)$} (t);
        \path (u) edge[-latex, color=red!50!black]
            node[font=\scriptsize] {$\varomega(u v)$} (v);
    \end{pgfonlayer}

    \node[shape=ellipse, fill=white, fit=(ds), opacity=0.8, path fading=circle with fuzzy edge 15 percent, inner sep=0.5pt] {};
    \node[shape=ellipse, fill=white, fit=(dt), opacity=0.8, path fading=circle with fuzzy edge 15 percent, inner sep=0.5pt] {};
\end{tikzpicture}

        \caption{Menor caminho de $s$ a $t$ que passa pela aresta $u v$.}
        \label{fig:ex2}
    \end{figure}

    ~
\end{minipage}

O pseudo-código assume que $k$ é finito. Caso $k$ possa ser irrestrito, seria necessário checar se $d_s[u] < \infty$ e $d_t[v] < \infty$ para garantir que o caminho $(s, \ldots, u, v, \ldots, t)$ existe.
