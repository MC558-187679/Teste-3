Prove que o algoritmo $\proc{Talvez-AGM}$ funciona ou mostre um contra-exemplo.
Justifique.

\begin{codebox}
    \Procname{$\proc{Talvez-AGM}(G, \omega)$}

    \li \Se $\abs{V[G]} \leq 2$ \Entao \Devolva $G$

    \li seja $V_1$, $V_2$ qualquer partição balanceada de $V[G]$
        \label{linha:ex1:divisao}
    \li seja $G_i$ o subgrafo completo de $G$ contendo apenas os vértices de $V_i$ e \\
    \qquad todas as arestas de $G$ ligando vértices de $V_i$ para $i=1,2$
    \li seja $\omega_i$ a restrição de $\omega$ a $G_i$ para $i=1,2$

    \li $T_1 \Recebe \proc{Talvez-AGM}(G_1, \omega_1)$
    \li $T_2 \Recebe \proc{Talvez-AGM}(G_2, \omega_2)$
    \li seja $e$ uma aresta de peso mínimo no corte $(V_1, V_2)$ em $G$

    \li \Devolva $T_1 \cup T_2 \cup \set{e}$
        \label{linha:ex1:combinacao}
\end{codebox}

\itemdsep
\newpage

O algoritmo nem sempre é capaz de encontrar uma AG \textbf{mínima}. Como os subgrafos $G_1$ e $G_2$ são arbitrários, a condição de serem balanceados não é o bastante para que a subestrutura ótima apareça. É possível que $G_1$ contenha, em um caso extremo, todas as arestas de maior peso do grafo $G$. Nessa situação, a AGM de $G_1$ não seria parte da AGM de $G$, sendo todas suas arestas trocadas (já que $G$ é completo) por arestas do corte $(V_1,V_2)$, em vez de apenas uma aresta de peso mínimo.

\begin{figure}[H]
    \centering
    \definecolor{grafo}{rgb}{0.373, 0.522, 0.761}
\begin{subfigure}[t]{0.48\textwidth}
    \centering

    \begin{tikzpicture}[
        scale=0.8, auto, node distance=2 cm and 2cm, on grid,
        semithick,
        state/.style={circle, color=white, draw, black, text=black, minimum width =1 cm}
    ]

        \node[state] (a) {a};
        \node[state] (b) [below left = of a] {b};
        \node[state] (c) [below right = of a] {c};

        \path (b) edge node[below, font=\footnotesize] {$\omega(bc) = 1$} (c);
        \path (a) edge node[above, font=\footnotesize, rotate=-45] {$\omega(ac) = 2$} (c);
        \path (a) edge node[above, font=\footnotesize, rotate=+45] {$\omega(ab) = 3$} (b);

        \node[draw, shape=ellipse, fit=(a)(b)(c), label={[color=grafo]$G$}, color=grafo] {};
    \end{tikzpicture}

    \caption{Entrada $G$ com $n = 3$.}
    \label{fig:ex1:Gexemplo}

\end{subfigure}\hfill%
\begin{subfigure}[t]{0.48\textwidth}
    \centering

    \begin{tikzpicture}[
        auto, node distance=2 cm and 2cm, on grid,
        semithick,
        state/.style={circle, color=white, draw, black, text=black, minimum width =1 cm}
    ]

        \node[state] (a) {a};
        \node[state] (b) [below left = of a] {b};
        \node[state] (c) [below right = of a] {c};

        \path (b) edge[opacity=0.2] node[below, font=\small, opacity=0.2] {$1$} (c);
        \path (a) edge[opacity=0.2] node[above right, font=\small, opacity=0.2] {$2$} (c);
        \path (a) edge node[above left, font=\small] {$3$} (b);

        \node[draw, shape=ellipse, fit=(a)(b), label={[color=grafo]$G_1$}, color=grafo] {};
        \node[draw, shape=ellipse, fit=(c), label={[color=grafo]$G_2$}, color=grafo] {};
    \end{tikzpicture}

    \caption{Possível partição $V_1$, $V_2$ balanceada.}
    \label{fig:ex1:particao}

\end{subfigure}
\\[8pt]
\begin{subfigure}[t]{0.48\textwidth}
    \centering\captionsetup{justification=centering}

    \begin{tikzpicture}[
        scale=0.8, auto, node distance=2 cm and 2cm, on grid,
        semithick,
        state/.style={circle, color=white, draw, black, text=black, minimum width =1 cm}
    ]

        \node[state] (a) {a};
        \node[state] (b) [below left = of a] {b};
        \node[state] (c) [below right = of a] {c};

        \path (b) edge node[below, font=\small] {$1$} (c);
        \path (a) edge[opacity=0.2] node[above right, font=\small, opacity=0.2] {$2$} (c);
        \path (a) edge node[above left, font=\small] {$3$} (b);

    \end{tikzpicture}

    \caption{Resultado (não-mínimo) de $\proc{Talvez-AGM}$, usando a partição proposta.}
    \label{fig:ex1:resultado}

\end{subfigure}\hfill%
\begin{subfigure}[t]{0.48\textwidth}
    \centering

    \begin{tikzpicture}[
        auto, node distance=2 cm and 2cm, on grid,
        semithick,
        state/.style={circle, color=white, draw, black, text=black, minimum width =1 cm}
    ]

        \node[state] (a) {a};
        \node[state] (b) [below left = of a] {b};
        \node[state] (c) [below right = of a] {c};


        \path (b) edge node[below, font=\small] {$1$} (c);
        \path (a) edge node[above right, font=\small] {$2$} (c);
        \path (a) edge[opacity=0.2] node[above left, font=\small] {$3$} (b);
    \end{tikzpicture}

    \caption{Árvore Geradora Mínima para $G$.}
    \label{fig:ex1:agminima}

\end{subfigure}


    \caption{Contra-exemplo para o algoritmo $\proc{Talvez-AGM}$.}
    \label{fig:ex1}
\end{figure}

Podemos ver na \cref{fig:ex1} como esse caso pode acontecer em um grafo $G$ completo de $n = 3$ vértices (\cref{fig:ex1:Gexemplo}). Se na etapa de Divisão (\textref{linha:ex1:divisao}{linha #1}) a partição for escolhida como $V_1 = \set{a, b}$ e $V_2 = \set{c}$, que é a balanceada, teremos $G_1$ e $G_2$ como na \cref{fig:ex1:particao}. No entanto, a AGM $T_1 = G_1$ de $G_1$ não faz parte da AGM de $G$, diferentemente do que foi assumido na etapa de Combinação (\textref{linha:ex1:combinacao}{linha #1}). Em uma situação como essa, o resultado de $\proc{Talvez-AGM}$ seria a árvore da \cref{fig:ex1:resultado}, que tem pesos maiores que a da \cref{fig:ex1:agminima}.
